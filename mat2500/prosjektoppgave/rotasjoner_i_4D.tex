\documentclass[a4paper,10pt,english]{article}
    \usepackage[utf8]{inputenc}
    \usepackage[norsk]{babel}
    % Standard stuff
    \usepackage{amsmath,graphicx,varioref,verbatim,amsfonts,geometry}
    % colors in text
    \usepackage{xcolor}
    % Hyper refs
    \usepackage[colorlinks]{hyperref}
    
    % Document formatting
    \setlength{\parindent}{0mm}
    \setlength{\parskip}{1.5mm}
    
    %Color scheme for listings
    \usepackage{textcomp}
    \definecolor{listinggray}{gray}{0.9}
    \definecolor{lbcolor}{rgb}{0.9,0.9,0.9}
    
    %Listings configuration
    \usepackage{listings}
    %Hvis du bruker noe annet enn python, endre det her for å få riktig highlighting.
    \lstset{
        backgroundcolor=\color{lbcolor},
        tabsize=4,
        rulecolor=,
        language=python,
        basicstyle=\scriptsize,
        upquote=true,
        aboveskip={1.5\baselineskip},
        columns=fixed,
        numbers=left,
        showstringspaces=false,
        extendedchars=true,
        breaklines=true,
        prebreak = \raisebox{0ex}[0ex][0ex]{\ensuremath{\hookleftarrow}},
        frame=single,
        showtabs=false,
        showspaces=false,
        showstringspaces=false,
        identifierstyle=\ttfamily,
        keywordstyle=\color[rgb]{0,0,1},
        commentstyle=\color[rgb]{0.133,0.545,0.133},
        stringstyle=\color[rgb]{0.627,0.126,0.941}
    }
            
    \newcounter{subproject}
    \renewcommand{\thesubproject}{\alph{subproject}}
    \newenvironment{subproj}{
        \begin{description}
        \item[\refstepcounter{subproject}(\thesubproject)]
    }{\end{description}}
    
    
    %opening
    \title{MAT2500 - Prosjektoppgave}
    \author{Jonas Folvik}
    
    \begin{document}
    
    \maketitle
    
    Rotasjons matrisen i 2 dimensjoner:
    
    $$\left(
    \begin{matrix}
        cos(\theta) & -sin(\theta) \\
        sin(\theta) &  cos(\theta)
    \end{matrix}
    \right)$$

    Rotasjons matrisen i 3 dimensjoner:

    $$\left(
    \begin{matrix}
        1 & 0 & 0 \\
        0 & cos(\theta) & -sin(\theta) \\
        0 & sin(\theta) &  cos(\theta)
    \end{matrix}
    \right)$$

    Rotasjons matrisen i 4 dimensjoner (enkel rotasjon):

    $$\left(
    \begin{matrix}
        1 & 0 & 0 & 0 \\
        0 & 1 & 0 & 0 \\
        0 & 0 & cos(\theta) & -sin(\theta) \\
        0 & 0 & sin(\theta) &  cos(\theta)
    \end{matrix}
    \right)$$

    Rotasjons matrisen i 4 dimensjoner (dobbel rotasjon):

    $$\left(
    \begin{matrix}
        cos(\phi) & -sin(\phi) & 0 & 0 \\
        sin(\phi) &  cos(\phi) & 0 & 0 \\
        0 & 0 & cos(\theta) & -sin(\theta) \\
        0 & 0 & sin(\theta) &  cos(\theta)
    \end{matrix}
    \right)$$

\end{document}
    