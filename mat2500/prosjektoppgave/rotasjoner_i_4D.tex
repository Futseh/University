\documentclass[a4paper,10pt,english]{article}
    \usepackage[utf8]{inputenc}
    \usepackage[norsk]{babel}
    % Standard stuff
    \usepackage{amsmath,graphicx,varioref,verbatim,amsfonts,geometry, amssymb}
    % colors in text
    \usepackage{xcolor}
    % Hyper refs
    \usepackage[colorlinks]{hyperref}
    \usepackage[backend = biber]{biblatex}
    \addbibresource{bibliography.bib}
    
    % Document formatting
    \setlength{\parindent}{0mm}
    \setlength{\parskip}{1.5mm}
    
    %Color scheme for listings
    \usepackage{textcomp}
    \definecolor{listinggray}{gray}{0.9}
    \definecolor{lbcolor}{rgb}{0.9,0.9,0.9}
    
    %Listings configuration
    \usepackage{listings}
    %Hvis du bruker noe annet enn python, endre det her for å få riktig highlighting.
    \lstset{
        backgroundcolor=\color{lbcolor},
        tabsize=4,
        rulecolor=,
        language=python,
        basicstyle=\scriptsize,
        upquote=true,
        aboveskip={1.5\baselineskip},
        columns=fixed,
        numbers=left,
        showstringspaces=false,
        extendedchars=true,
        breaklines=true,
        prebreak = \raisebox{0ex}[0ex][0ex]{\ensuremath{\hookleftarrow}},
        frame=single,
        showtabs=false,
        showspaces=false,
        showstringspaces=false,
        identifierstyle=\ttfamily,
        keywordstyle=\color[rgb]{0,0,1},
        commentstyle=\color[rgb]{0.133,0.545,0.133},
        stringstyle=\color[rgb]{0.627,0.126,0.941}
    }
            
    \newcounter{subproject}
    \renewcommand{\thesubproject}{\alph{subproject}}
    \newenvironment{subproj}{
        \begin{description}
        \item[\refstepcounter{subproject}(\thesubproject)]
    }{\end{description}}
    
    
    %opening
    \title{MAT2500 - Prosjektoppgave}
    \author{Jonas Folvik}
    
    \begin{document}
    
    \maketitle

    Når vi tenker på rotasjoner så kan man forestille seg et ark eller en mynt som snurrer flatt på et bord. Dette er eksempler på rotasjoner i to dimensjoner.
    Vi kan også forestille seg hjulene på en bil i bevegelse eller jorden som går rundt sola, dette er da eksempler på rotasjoner i tre dimensjoner.
    Hvis arket eller mynten som snurrer står stille på samme sted så blir et punkt, et 0 dimensjonalt objekt, holdt fast. Mens i tre dimensjonale rotasjoner så blir en linje, et 1 dimensjonalt objekt, holdt fast.
    På grunn av dette kan man finne ut at ved en rotasjon i et $n$ dimensjonalt system, så vil man rotere rundt $n-2$ dimensjonale objekter.
    I den fjerde dimensjonen vil man da rotere rundt to dimensjonale objekter, bedre kjent som plan.

    Rotasjons matrisen i to dimensjoner kan se slik ut:

    $$
    \left(
    \begin{matrix}
        \cos(\theta) & -\sin(\theta) \\
        \sin(\theta) &  \cos(\theta)
    \end{matrix}
    \right)
    $$

    Denne vil da holde fast, fiksere, punktet $(0, 0)$, bedre kjent som Origo, mens alle andre punkter og vektorer, i planet, vil rotere med en vinkel $\theta$.
    
    Et eksempel på en rotasjons matrise i tre dimensjoner er:

    $$
    \left(
    \begin{matrix}
        1 & 0 & 0 \\
        0 & \cos(\theta) & -\sin(\theta) \\
        0 & \sin(\theta) &  \cos(\theta)
    \end{matrix}
    \right)
    $$

    Denne vil fiksere linjen som ligger på $x_{1}$-aksen, mens alle andre punkter og vektorer, i rommet, vil rotere med vinkel $\theta$

    \textbf{Teorem 1} \textit{Gitt et Euklidisk rom E av dimensjon n, for hver ortogonal lineær transformasjon $f \colon E \rightarrow E$ fins det en ortonormal basis $\left( e_{1}, \dots, e_{n} \right)$
                              slik at matrisen for $f$ med hensyn på denne basisen er en blokk diagonal matrise på formen: 
                              $$
                              \begin{pmatrix}
                                A_{1}  &        & \cdots &        \\
                                       & A_{2}  & \cdots &        \\
                                \vdots & \vdots & \ddots & \vdots \\
                                       &        & \cdots & A_{p}
                              \end{pmatrix}
                              $$
                              slik at hver blokk $A_{i}$ er enten $1$, $-1$ eller en to-dimensjonal matrise på formen:
                              $$ A_{i} = 
                              \begin{pmatrix}
                                \cos(\theta_{i}) & -\sin(\theta_{i}) \\
                                \sin(\theta_{i}) &  \cos(\theta_{i})
                              \end{pmatrix}
                              $$
                              hvor $0 < \theta_{i} < \pi$. In particular, egenverdiene av $f_{\mathbb{C}}$ er på formen: $\cos\theta_{i} \pm i\sin\theta_{i}$, $1$, eller $-1$.} \cite{GMA}

    \textit{Proof.} Tilfellet når $n = 1$ er trivielt. Som i beviset for teorem 11.2.9 \cite{GMA}, $f_{\mathbb{C}}$ har en egenverdi $z = \lambda+i\mu$, hvor $\lambda$, $\mu \in \mathbb{R}$.
                    Siden $f \circ f^{*} = f^{*} \circ f = id$, transformasjonen $f$ er invertibel. Faktisk så har egenverdiene til $f$ en absolutt verdi lik $1$, $\lvert z \rvert = 1$.
                    Hvis $z \in \mathbb{C}$ er en egenverdi for $f$, og $u$ er en egenvektor for $z$, har vi:
                    $$
                    \langle f(u), f(u) \rangle = \langle zu, zu \rangle = z \bar{z} \langle u, u \rangle
                    $$
                    og
                    $$
                    \langle f(u), f(u) \rangle = \langle u, (f^{*} \circ f)(u) \rangle = \langle u, u \rangle
                    $$
                    fra dette så får vi at:
                    $$
                    z \bar{z} \langle u, u \rangle = \langle u, u \rangle
                    $$

                    Siden $u \neq 0$, har vi $z \bar{z} = 1$, som vil si at $\lvert z \rvert = 1$. Som en konsekvens av dette så er egenverdiene av $f_{\mathbb{C}}$ på formen: $\cos\theta_{i} \pm i\sin\theta_{i}$, $1$, eller $-1$.
                    

    Et eksempel på en rotasjons matrise i 4 dimensjoner er:

    $$
    R_{\phi, \theta} = \left(
    \begin{matrix}
        \cos(\phi) & -\sin(\phi) & 0 & 0 \\
        \sin(\phi) &  \cos(\phi) & 0 & 0 \\
        0 & 0 & \cos(\theta) & -\sin(\theta) \\
        0 & 0 & \sin(\theta) &  \cos(\theta)
    \end{matrix}
    \right)
    $$

    Denne typen rotasjoner blir ofte kalt for en dobbel rotasjon, da man roterer rundt to plan.
    I dette eksempelet roterer vi om $x_{1}x_{2}$-planet med vinkel $\phi$ og vi roterer om $x_{3}x_{4}$-planet med vinkel $\theta$
    Hvis $\phi = \theta \neq 0$ i en dobbel rotasjon så kalles det for en isoklinisk rotasjon. Hvis vi setter $\phi = 0$ i dobbel rotasjonen så får vi matrisen for $R_{\theta}$
    Når vi setter $\phi = 0$ så får vi matrisen:

    $$
    R_{\theta} = \left(
    \begin{matrix}
        1 & 0 & 0 & 0 \\
        0 & 1 & 0 & 0 \\
        0 & 0 & \cos(\theta) & -\sin(\theta) \\
        0 & 0 & \sin(\theta) &  \cos(\theta)
    \end{matrix}
    \right)
    $$

    Denne rotasjonen fikserer $x_{1}x_{2}$-planet og roterer $x_{3}x_{4}$-planet med vinkel $\theta$. Dette er et eksempel på en rotasjon som kalles for en enkel rotasjon.

    For at en ortogonal operator T skal være en rotasjon, så må den være orienteringsbevarende. Det vil si at: $det(T) = 1$

    Sjekker om det stemmer for en dobbel rotasjon og en enkel rotasjon. Bruker $R_{\phi, \theta}$ og $R_{\theta}$:
    $$
    \det(R_{\phi, \theta}) = \cos(\phi)^{2}(\cos(\theta)^{2} + \sin(\theta)^{2}) + \sin(\phi)^{2}(\cos(\theta)^{2} + \sin(\theta)^{2}) = \cos(\phi)^{2} + \sin(\phi)^{2} = 1
    $$
    $$
    \det(R_{\theta}) = \cos(\theta)^{2} + \sin(\theta)^{2} = 1
    $$    

\printbibliography

\end{document}