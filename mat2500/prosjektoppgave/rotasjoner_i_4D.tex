\documentclass[a4paper,10pt,english]{article}
    \usepackage[utf8]{inputenc}
    \usepackage[norsk]{babel}
    % Standard stuff
    \usepackage{amsmath,graphicx,varioref,verbatim,amsfonts,geometry, amssymb}
    % colors in text
    \usepackage{xcolor}
    % Hyper refs
    \usepackage[colorlinks]{hyperref}
    \usepackage[backend = biber]{biblatex}
    \addbibresource{bibliography.bib}
    
    % Document formatting
    \setlength{\parindent}{0mm}
    \setlength{\parskip}{1.5mm}
    
    %Color scheme for listings
    \usepackage{textcomp}
    \definecolor{listinggray}{gray}{0.9}
    \definecolor{lbcolor}{rgb}{0.9,0.9,0.9}
    
    %Listings configuration
    \usepackage{listings}
    %Hvis du bruker noe annet enn python, endre det her for å få riktig highlighting.
    \lstset{
        backgroundcolor=\color{lbcolor},
        tabsize=4,
        rulecolor=,
        language=python,
        basicstyle=\scriptsize,
        upquote=true,
        aboveskip={1.5\baselineskip},
        columns=fixed,
        numbers=left,
        showstringspaces=false,
        extendedchars=true,
        breaklines=true,
        prebreak = \raisebox{0ex}[0ex][0ex]{\ensuremath{\hookleftarrow}},
        frame=single,
        showtabs=false,
        showspaces=false,
        showstringspaces=false,
        identifierstyle=\ttfamily,
        keywordstyle=\color[rgb]{0,0,1},
        commentstyle=\color[rgb]{0.133,0.545,0.133},
        stringstyle=\color[rgb]{0.627,0.126,0.941}
    }
            
    \newcounter{subproject}
    \renewcommand{\thesubproject}{\alph{subproject}}
    \newenvironment{subproj}{
        \begin{description}
        \item[\refstepcounter{subproject}(\thesubproject)]
    }{\end{description}}
    
    
    %opening
    \title{MAT2500 - Prosjektoppgave}
    \author{Jonas Folvik}
    
    \begin{document}
    
    \maketitle
    
    I 2 dimensjoner roterer vi om et 0 dimensjonalt objekt, et punkt. I 3 dimensjoner roterer vi om et 1 dimensjonalt objekt, en linje.
    I 4 dimensjoner roterer vi om et 2 dimensjonalt objekt, et plan. Det er også mulig å rotere rundt 2 plan, bedre kjent som en dobbel rotasjon.

    En av matrisene for en enkel rotasjon, hvor man roterer om 1 plan og holder 1 plan fast, kan se slik ut:

    $$R_{\theta} = \left(
    \begin{matrix}
        1 & 0 & 0 & 0 \\
        0 & 1 & 0 & 0 \\
        0 & 0 & cos(\theta) & -sin(\theta) \\
        0 & 0 & sin(\theta) &  cos(\theta)
    \end{matrix}
    \right)$$

    En av matrisene for en dobbel rotasjon, hvor man roterer rundt 2 plan, kan se slik ut:

    $$
    R_{\phi, \theta} = \left(
    \begin{matrix}
        cos(\phi) & -sin(\phi) & 0 & 0 \\
        sin(\phi) &  cos(\phi) & 0 & 0 \\
        0 & 0 & cos(\theta) & -sin(\theta) \\
        0 & 0 & sin(\theta) &  cos(\theta)
    \end{matrix}
    \right)$$

    Hvis $\phi = \theta \neq 0$ i en dobbel rotasjon så kalles det for en isoklinisk rotasjon. Hvis vi setter $\phi = 0$ i dobbel rotasjonen så får vi matrisen for $R_{\theta}$

    For at en ortogonal operator T skal være en rotasjon, så må den være orienteringsbevarende. Det vil si at: $det(T) = 1$

    Sjekker om det stemmer for både en enkel rotasjon og en dobbel rotasjon.

    Enkel rotasjon:
    $$
    det(R_{\theta}) = cos(\theta)^{2} + sin(\theta)^{2} = 1
    $$

    Dobbel rotasjon:
    $$
    det(R_{\phi, \theta}) = cos(\phi)^{2}(cos(\theta)^{2} + sin(\theta)^{2}) + sin(\phi)^{2}(cos(\theta)^{2} + sin(\theta)^{2}) = cos(\phi)^{2} + sin(\phi)^{2} = 1
    $$

    Egenverdiene til en dobbel rotasjon er:
    
    \begin{align*}
        cos(\phi)+isin(\phi) &= e^{i\phi} \nonumber \\
        cos(\phi)-isin(\phi) &= e^{-i\phi} \nonumber \\
        cos(\theta)+isin(\theta) &= e^{i\theta} \nonumber \\
        cos(\theta)-isin(\theta) &= e^{-i\theta} \nonumber
    \end{align*}

    En operator $T$ er i $SO(4)$ hvis og bare hvis $T$ er en rotasjon.

    Viser det for enkel rotasjoner.

    I en bestemt basis er matrisen til en enkel rotasjon gitt ved $R_{\theta}$ som er ortogonalk med determinant 1. Dvs at alle enkel rotasjoner er i $SO(4)$.
    For å vise den andre implikasjonen la $A$ være standarmatrisen til $T$. Da er $A^{t}A = I$ og $det A = det A^{t} = 1$. Vi vil først vise at $A$ har 1 som egenverdi, som er det samme som å vise 
    at $det(A - I) = 0$. Merk at $A^{t}(A - I) = (I - A^{t}) = (I - A)^{t}$. Vi har 

    $$ det(A- I) = det(A^{t})det(A - I) = det A^{t}(A - I) = det(I- A^{t}) = det(-(A^{t} - I)) $$

    Men hvis $B$ er en $4 \times 4$ matrise så er $det(-B) = -det B$. Derfor må $det(A - I) = 0$.
    
    Velg en egenvekor $v_{1}$ for egenverdien 1 med $||v_{1}|| = 1$. Velg en ortonormal basis $\{v_{2}, v_{3}, v_{4}\}$ for rommet gjennom origo som er ortogonal til $v_{1}$.
    Da blir $\{v_{1}, v_{2}, v_{3}, v_{4}\}$ en ortonormal basis for $\mathbb{R}^{4}$ og hvis $P$ er matrisen med søyler $v_{1}, v_{2}, v_{3}, v_{4}$ så er $P$ ortogonal.
    Da må $A' = P^{-1}AP$ også være ortogonal med $det A' = det A = 1$.
    Nå er 

    $$
    A = \left(
    \begin{matrix}
        1 & 0 & 0 & 0 \\
        0 & a & b & c \\
        0 & d & e & f \\
        0 & g & h & i
    \end{matrix}
    \right)
    $$

Der $\left( \begin{matrix} a & b & c \\ d & e & f \\ g & h & i \end{matrix} \right)$ må være en ortogonal $3 \times 3$ matrise med determinant 1. Men da sier Setning 3.4 \cite{kompendium} at

$$
\left(
    \begin{matrix}
        a & b & c \\
        d & e & f \\
        g & h & i
    \end{matrix}
\right)
=
\left(
    \begin{matrix}
        1 & 0 & 0 \\
        0 & cos(\theta) & -sin(\theta) \\
        0 & sin(\theta) &  cos(\theta)
    \end{matrix}
\right)
$$
for en $\theta$ og vi har vist at $T$ er en enkel rotasjon.

Viser det for en dobbel rotasjon.

\printbibliography

\end{document}